\documentclass[a4paper, 12pt]{memoir}

\raggedbottom

\isopage
\checkandfixthelayout

  
\normalfont
\fixpdflayout

% Palatino
\fontfamily{ppl}\selectfont

\chapterstyle{default}
\chapterstyle{culver}

\begin{document}

\frontmatter
\mainmatter

\chapter*{PROLOGUE}
\thispagestyle{plain}
Years ago, when I was very small, we lived in a great house in a long, straight, brown-coloured street, in the east end of London.  It was a noisy, crowded street in the daytime; but a silent, lonesome street at night, when the gas-lights, few and far between, partook of the character of lighthouses rather than of illuminants, and the tramp, tramp of the policeman on his long beat seemed to be ever drawing nearer, or fading away, except for brief moments when the footsteps ceased, as he paused to rattle a door or window, or to flash his lantern into some dark passage leading down towards the river.

The house had many advantages, so my father would explain to friends who expressed surprise at his choosing such a residence, and among these was included in my own small morbid mind the circumstance that its back windows commanded an uninterrupted view of an ancient and much-peopled churchyard.  Often of a night would I steal from between the sheets, and climbing upon the high oak chest that stood before my bedroom window, sit peering down fearfully upon the aged gray tombstones far below, wondering whether the shadows that crept among them might not be ghosts---soiled ghosts that had lost their natural whiteness by long exposure to the city’s smoke, and had grown dingy, like the snow that sometimes lay there.

I persuaded myself that they were ghosts, and came, at length, to have quite a friendly feeling for them.  I wondered what they thought when they saw the fading letters of their own names upon the stones, whether they remembered themselves and wished they were alive again, or whether they were happier as they were.  But that seemed a still sadder idea.

One night, as I sat there watching, I felt a hand upon my shoulder.  I was not frightened, because it was a soft, gentle hand that I well knew, so I merely laid my cheek against it.

``What's mumma's naughty boy doing out of bed? Shall I beat him?''  And the other hand was laid against my other cheek, and I could feel the soft curls mingling with my own.

``Only looking at the ghosts, ma,'' I answered. ``There's such a lot of 'em down there.''  Then I added, musingly, ``I wonder what it feels like to be a ghost.''

My mother said nothing, but took me up in her arms, and carried me back to bed, and then, sitting down beside me, and holding my hand in hers---there was not so very much difference in the size---began to sing in that low, caressing voice of hers that always made me feel, for the time being, that I wanted to be a good boy, a song she often used to sing to me, and that I have never heard any one else sing since, and should not care to.

But while she sang, something fell on my hand that caused me to sit up and insist on examining her eyes.  She laughed; rather a strange, broken little laugh, I thought, and said it was nothing, and told me to lie still and go to sleep.  So I wriggled down again and shut my eyes tight, but I could not understand what had made her cry.
\backmatter

\end{document}
